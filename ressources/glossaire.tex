
\begin{comment}
\newglossaryentry{id}
{
    name=str_affiche,
    description={
    
    }
}
\end{comment}

\newglossaryentry{criticite}
{
    name=criticité,
    description={
    La détermination et la hiérarchisation du degré d'importance et de la disponibilité d'un système d'information.
    }
}

\newglossaryentry{dcp}
{
    name=DCP,
    description={
    Toute information relative à une personne physique susceptible d'être identifiée, directement ou indirectement
    }
}


\newglossaryentry{pseudonymisation}
{
    name=pseudonymisation,
    description={
    La pseudonymisation est un procédé de remplacement des données permettant d’identifier une personne physique par d’autres données. Cette technique permet de protéger les données personnelles de l’individu concerné. Contrairement à l’anonymisation, la pseudonymisation est un procédé réversible permettant de retrouver la trace des données remplacées. 
    }
}

\newglossaryentry{anonymisation}
{
    name=anonymisation,
    description={
    L’anonymisation est un traitement qui consiste à utiliser un ensemble de techniques de manière à rendre impossible, en pratique, toute identification de la personne par quelque moyen que ce soit et de manière irréversible.
    }
}

\newglossaryentry{donnes_personnelles}
{
    name=données personelles,
    description={
    Une données personnelle est "toute information se rapportant à une personne physique identifiée ou identifiable. Une personne peut être identifiée directement ( par son nom, son prénom ) ou indirectement ( téléphone, mail, données biométriques, voix, image ... )   
    }
}
 
\newglossaryentry{rgpd}
{
    name=RGPD,
    description={
    Le règlement général de protection des données (RGPD) est un texte réglementaire européen qui encadre le traitement des données de manière égalitaire sur tout le territoire de l’Union européenne (UE). Il est entré en application le 25 mai 2018.
    Le RGPD s’inscrit dans la continuité de la loi française « Informatique et Libertés » de 1978, modifiée par la loi du 20 juin 2018 relative à la protection des données personnelles, établissant des règles sur la collecte et l’utilisation des données sur le territoire français.
    }
}




\newglossaryentry{h_index}
{
    name=h-index,
    description={
    Le h-index (ou facteur h), créé par le physicien Jorge Hirsch en 2005, est un indicateur d’impact des publications d’un chercheur. Il prend en compte le nombre de publications d’un chercheur et le nombre de leurs citations. Le h-index d’un auteur est égal au nombre h le plus élevé de ses publications qui ont reçu au moins h citations chacune.
    }
}


\newglossaryentry{bibliometrie}
{
    name=bibliométrie,
    description={
    La bibliométrie est l’application de méthodes statistiques et mathématiques pour mesurer et évaluer la production et la diffusion de publications. Elle génère des formules, parfois sophistiquées, visant à donner un indice de performance de la recherche pour un chercheur ou une chercheuse, un laboratoire, un établissement, un pays, etc.
    source:\url{https://bu.univ-larochelle.fr/lappui-a-la-recherche/valorisation-de-la-recherche/bibliometrie-et-impact-de-la-recherche-2/}
    }
}

\newglossaryentry{entrepot_ddr}
{
    name=Entrepôt de données de la recherche,
    description={
    Un entrepôt de données de recherche (Research Data Repository ou Data Repository) est une base de données destinée à accueillir, conserver, rendre visibles et accessibles des données de recherche. Son rôle est de permettre le dépôt ou la collecte de données, leur description, leur accès, et leur partage en vue de leur réutilisation. Chaque entrepôt dispose généralement d’une politique de dépôt, de description et de diffusion des données. 
    }
}


\newglossaryentry{data_paper}
{
    name=Data Paper,
    description={
    Un data paper est une publication scientifique qui décrit précisément un jeu de données, et informe la communauté scientifique de son existence, de ses modalités et de son potentiel de réutilisation.
    }
}

\newglossaryentry{tei}
{
    name=TEI,
    description={
    La Text Encoding Initiative (abrégé en TEI, en français « initiative pour l’encodage du texte ») est un format de balisage et une communauté académique internationale dans le champ des humanités numériques visant à définir des recommandations pour l’encodage de ressources numériques, et plus particulièrement de documents textuels.
    }
}


\newglossaryentry{nakala}
{
    name=Nakala,
    description={
    NAKALA est un entrepôt de données dont le but est de préserver et de disséminer les données produites par les productions des projets de recherche français en Sciences Humaines et Sociales dans le respect des principes FAIR (Cf \url{https://documentation.huma-num.fr/nakala-faq/}). NAKALA est destiné en premier lieu aux projets de recherche des institutions ayant une affiliation dépendante du MESR (Ministère de l’Enseignement Supérieur et de la Recherche).
    }
}


\newglossaryentry{indexation}
{
    name=indexation,
    description={
    Attribution à un document de termes distinctifs (des mots-clés par exemple) renseignant sur son contenu et permettant de le retrouver.
    source : Ouvrir la science
    }
}
\newglossaryentry{doi}
{
    name=DOI,
    description={
    Le digital object identifier (DOI, littéralement « identifiant numérique d'objet») est un mécanisme d'identification de ressources stable, qui peuvent être des ressources numériques, comme un film, un rapport, des articles scientifiques, ainsi que des personnes ou tout autre type d'objet.\\
    source : \url{https://fr.wikipedia.org/wiki/Digital_Object_Identifier}
    }
}
\newglossaryentry{pid}
{
    name=PID,
    description={
    Un identifiant pérenne (Persistent identifier ou PID) est un identifiant qui est assigné à un objet de façon permanente. Il est disponible et gérable à long terme ; il ne changera pas si l'objet est renommé ou déplacé (changement de site, d'entrepôts de données...).\\ 
    source : \url{https://www.inist.fr/wp-content/uploads/donnees/co/module_Donnees_recherche_37.html}
    }
}
\newglossaryentry{scienceouverte}
{
    name=Science Ouverte,
    description={
    La science ouverte est la diffusion sans entrave des publications et des données de la recherche. Elle s’appuie sur l’opportunité que représente la mutation numérique pour développer l’accès ouvert aux publications et – autant que possible – aux données de la recherche.
    }
}
\newglossaryentry{ddr}
{
    name= données de la recherche,
    description={
    Les données de la recherche sont définies comme des enregistrements factuels (chiffres, textes, images et sons), qui sont utilisés comme sources principales pour la recherche scientifique et sont généralement reconnus par la communauté scientifique comme nécessaires pour valider des résultats de recherche. (recherche.data.gouv)
                }
}

\newglossaryentry{corpus}
{
    name=corpus,
    description={
                Ensemble de textes établi selon un principe de documentation exhaustive, un critère thématique ou exemplaire en vue de leur étude linguistique
                }
}
\newglossaryentry{pgd}
{
    name=PGD,
    description={
                Document dont la rédaction doit être initiée au commencement d'un projet de recherche, décrivant les données et comment elles seront partagées et conservées pendant et après le projet
                }
}

\newglossaryentry{fair}
{
    name=FAIR,
    description={
                Les principes FAIR (Findable, Accessible, Interoperable, Reusable) décrivent comment les données doivent être organisées pour être plus facilement accessibles, comprises, échangeables et réutilisables
                }
}
\newglossaryentry{datapapers}
{
    name=Data Papers,
    description={
    Le data paper (ou data article) est une publication scientifique dont le but principal est de décrire un ou plusieurs jeux de données, plutôt que des résultats d'analyse
    }
}
\newglossaryentry{metadonnees}
{
    name=métadonnées,
    description={
    Les métadonnées sont utiles pour exploiter un jeu de
données produit par d'autres. Ce sont les nombreuses
informations relatives au contexte de production, à la
méthodologie, à la description du jeu, etc.
Les entrepôts de données sont des espaces qui rendent
accessibles les jeux de données et les métadonnées qui
y sont associées.
Les entrepôts en auto-dépôt sont des espaces de
partage libre de données et sans vérification de la
qualité des métadonnées. (CNRS)
    }
}

\newglossaryentry{sauvegarde}
{
    name=sauvegarde,
    description={
    Une copie de tout ou une partie des fichiers sur un système séparé des données originelles, à des fins de récupération sur le court terme en cas de perte ou de dégradation des données. Il s’agit d’une image figée dans le temps des fichiers ; la fréquence des sauvegardes et le nombre de versions conservées simultanément dépendent des outils, services et besoins.
    Source : \url{https://bu.univ-lille.fr/chercheurs-doctorants/science-ouverte/donnees-de-recherche/sauvegarde-et-stockage-des-donnees}
    }
}

\newglossaryentry{archive}
{
    name=archives,
    description={
    Une organisation dont la mission est la conservation des informations afin d’assurer l’accès et l’utilisation par une communauté spécifique, ou un site où des données lisibles par machine sont stockées, conservées et éventuellement redistribuées aux personnes intéressées à utiliser lesdites données.
    Source : \url{https://bu.univ-lille.fr/chercheurs-doctorants/science-ouverte/donnees-de-recherche/sauvegarde-et-stockage-des-donnees}
    }
}

\newglossaryentry{licence_de_diffusion}
{
    name = licence de diffusion,
    description ={
    Une licence de diffusion est un instrument juridique, complémentaire au droit d’auteur. Elle permet au titulaire des droits sur une œuvre d’accorder à l’avance aux utilisateurs certains droits d’utilisation de cette œuvre. Elle préserve les droits moraux de l’auteur en imposant toujours l’obligation d’attribution (citation de la source).
    Source : \url{https://coop-ist.cirad.fr/etre-auteur/utiliser-les-licences-creative-commons/2-qu-est-ce-qu-une-licence-de-diffusion}
    }
}

\newglossaryentry{tal}
{
    name = TAL,
    description={
    Le traitement automatique des langues, en anglais natural language processing ou NLP, est un domaine multidisciplinaire impliquant la linguistique, l'informatique et l'intelligence artificielle, qui vise à créer des outils de traitement du langage naturel pour diverses applications.
    }
}

\newglossaryentry{tokenisation}
{
    name=Tokenisation,
    description={
    La tokenisation est une étape fondamentale du traitement du langage naturel (NLP). Elle consiste à découper un texte en unités plus petites, appelées tokens, qui peuvent ensuite être traitées par des modèles d'apprentissage automatique.
    }
}

\newglossaryentry{pos_tagging}
{
    name=POS Tagging,
    description={
    En linguistique, l'étiquetage morpho-syntaxique (aussi appelé étiquetage grammatical, POS tagging (part-of-speech tagging) en anglais) est le processus qui consiste à associer aux mots d'un texte les informations grammaticales correspondantes comme la partie du discours, le genre, le nombre, etc. à l'aide d'un outil informatique
    }
}

\newglossaryentry{ner}
{
    name=NER,
    description={
    La reconnaissance d'entités nommées est une sous-tâche de l'activité d'extraction d'information dans des corpus documentaires. Elle consiste à rechercher des objets textuels (c'est-à-dire un mot, ou un groupe de mots) catégorisables dans des classes telles que noms de personnes, noms d'organisations ou d'entreprises, noms de lieux, quantités, distances, valeurs, dates, etc. 
    }
}

\newglossaryentry{spacy}
{
    name=spaCy,
    description={
    SpaCy est une bibliothèque logicielle Python de traitement automatique des langues développée par Matt Honnibal et Ines Montani. SpaCy est un logiciel libre publié sous licence MIT
    }
}

\newglossaryentry{nltk}
{
    name=NLTK,
    description={
    Natural Language Toolkit est une bibliothèque logicielle en Python permettant un traitement automatique des langues, développée par Steven Bird et Edward Loper du département d'informatique de l'université de Pennsylvanie.
    }
}


\newglossaryentry{tei_header}
{
    name=teiHeader,
    description={
     L'en-tête TEI (teiHeader) fournit des informations descriptives et déclaratives qui constituent une page de titre électronique au début de tout texte conforme à la TEI.
    }
}

\newglossaryentry{langage_de_balisage}
{
    name=langages de balisage,
    description={
    En informatique, les langages de balisage représentent une classe de langages spécialisés dans l'enrichissement d'information textuelle. Ils utilisent des balises, unités syntaxiques délimitant une séquence de caractères ou marquant une position précise à l'intérieur d'un flux de caractères.
    }
}

\newglossaryentry{xml}
{
    name=XML,
    description={
    L'Extensible Markup Language, généralement appelé XML, « langage de balisage extensible » en français, est un métalangage informatique de balisage générique qui est un sous-ensemble du Standard Generalized Markup Language. 
    }
}


\newglossaryentry{tmx}
{
    name=TMX,
    description={
    Translation Memory eXchange est une spécification XML pour l'échange de données de mémoire de traduction entre des outils de traduction et de localisation assistés par ordinateur avec peu ou pas de perte de données critiques.
    }
}

\newglossaryentry{lmf}
{
    name=LMF,
    description={
    Lexical Markup Framework (LMF ou cadre de balisage lexical, en français) est le standard de l'Organisation internationale de normalisation (plus spécifiquement au sein de l'ISO/TC37) pour les lexiques du traitement automatique des langues (TAL).
    }
}

\newglossaryentry{token}
{
    name=token,
    description={
    Un token est une séquence de caractères dans un document particulier qui sont regroupés en tant qu'unités sémantique utile pour le traitement.
    }
}