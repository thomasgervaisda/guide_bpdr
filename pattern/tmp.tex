
\documentclass{book}
    % General document formatting
    \usepackage[margin=0.7in]{geometry}
    \usepackage[parfill]{parskip}
    \usepackage[utf8]{inputenc}
    \usepackage{url}
    % Related to math
    \usepackage{amsmath,amssymb,amsfonts,amsthm}
    \usepackage{comment}
    \usepackage{glossaries}
    \usepackage{biblatex}

%Bibliographie
\bibliography{ressources/biblio.bib}

%Index
\makeindex 

%Glossaire
\makenoidxglossaries
\loadglsentries{ressources/glossaire}


\begin{document}



\begin{comment}
\begin{enumerate}
	\item Grand-1
        \begin{enumerate}
            \item Petit-a
            \item Petit-b
            \item Petit-c
        \end{enumerate}
	\item Grand-2
        \begin{enumerate}
            \item Petit-a
            \item Petit-b
            \item Petit-c
        \end{enumerate}
	\item Grand-3
        \begin{enumerate}
            \item Petit-a
            \item Petit-b
            \item Petit-c
        \end{enumerate}
\end{enumerate}

\begin{itemize}
    \item [Liens]
        \url{http://...}\\
        \url{http://...}\\
        \url{http://...}\\
    \item [Mots clé]
        \gls{Mot1},\gls{Mot2}
\end{itemize}
\end{comment}

\printnoidxglossaries

\end{document}